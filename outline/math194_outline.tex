\documentclass{article}[12pt]

\usepackage{amsmath}

\title{
    Playing Optimal Go Endgames with Combinatorial Game Theory \\
    \large Outline
}
\date{}
\author{Anders Poirel | apoirel@ucsc.edu}

\begin{document}
    
\maketitle

\begin{abstract}
    Combinatorial game theory studies deterministic, perfect-information 
    games with guaranteed termination. As a perfect-information, symmetric 
    two-player game, Go is one of the canonical objects of study in the field.
    While the combinatorial complexity of the game is such that it cannot be fully solved,
    Go endgames can be decomposed into simpler objects where combinatorial game theory can 
    be used to achieve optimal play. With the small victory margins observed at high
    levels of (human) play, using results from combinatorial game theory can have a 
    noticeable impact on one's odds of winning 
\end{abstract}

\tableofcontents

\section{Game of Go}

\subsection{Background}

\subsection{Rules of Go}
Review of the American Go Association rules for the game. 
Why the traditional japanese rules leave ambiguities making the 
mathematical analysis of the game 

\section{Combinatorial Game Theory for Go}

This sections goes over a few fundamental concepts of combinatorial game 
theory, illustrating each with examples of Go. Concepts that do not apply 
to the analysis of go endgames (e.g. cardinal numbers) are glossed over.

\subsection{Basic Concepts}
Review of Conway's definition of Game. Fundamental structures and 
concepts

\subsubsection{Mathematical Re-formulation of Rules of Go}
Detail on how this definition is a natural fit for the Game of go. 
Re-formulation of rules of game for more convenient analysis 
(while showing their quasi-equivalence with AGA rules). 

\subsection{Conway Induction}
Fundamental theorems. Conway induction as the main tool of proof in the 
theory

\subsection{Arithmetic of Games}
Games as an algebraic group/ring/field. Illustration of the additive 
structure of the game of Go

\subsection{Simplifying Games}

Some useful theorems that can be used to simplify positions 

\subsection{Surreal Numbers}
The next two subsections explore some sub-classes of Conway Numbers 
(also known as surreal numbers) and how they can be used to evaluate 
certain moves in Go.

\subsubsection{Integers and Fractions}

Integers as finished game in normal rules. Fractional point value 
of endgame plays.

\subsubsection{Inifinitesimals}

Infinitesimals as moves with smaller than fractional value. 
Preview of the concept of chilling.

\section{Optimal Go Endgame Strategy}

This section uses concepts from the previous section to derive exact 
results for optimal play in the endgame.

\subsection{Endgame Positions}
Type of positions where the theory is useful in deriving exact results.

\subsection{Chilling}
Concepts of chilling and ``hot'' moves.

\subsection{Corridors}
Solving positions involving simple shapes.

\subsection{More Complicated Shapes}
Examples of applications of the theory to solving endgame positions 
involving more complex shapes. 

\section{Conclusion}
Advantages of the theory over other approaches such as naive tree search and machine 
learning for, especially from a tractability/exactness perspective. Limits of the
theory (e.g. positions involving KOs), and directions for future reading. 
Some important open research problems in the theory of Go. 

\bibliographystyle{ieeetran}

\bibliography{references}

\end{document}
