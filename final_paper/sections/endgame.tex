\documentclass[../math194_paper.tex]{subfiles}

\begin{document}

This sections covers some applications of the combinatorial game theory we 
described to solving a class of endgame Go positions. In particular, these
will typically be positions where optimal play can yield an net advantage 
of one point.

\subsection{Endgame Positions}

\begin{definition}[Parity of positions] \label{parity} \:
    \begin{itemize}
        \item A Go position is \textit{even} if the number of empty intersections plus the number of 
        captured prisoners is even
        \item A Go position is odd if it is not even
    \end{itemize}
\end{definition}
Parities add as expected and alternate during play.

\begin{definition}[Elementary position]
    \label{elementary}
    A Go position is \textit{elementary} if when completely played out, every point on the board 
    is either occupied by a stone in an \textit{alive} group, or becomes territory for a player.
\end{definition}
This implies that positions that where the ownership of territory or the life of groups are undecided
are excluded from our analysis. This includes \textit{kos} and \textit{sekis}.

We will require that Go positions we analyze are both even and elementary,
since the theorems we will use for this in the following section require these assumptions to function.

\end{document}