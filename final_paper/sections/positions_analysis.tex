\documentclass[../math194_paper.tex]{subfiles}

\begin{document}

\subsection{Tying Everything Together}

We can then use the following approach to determine the outcome of an endgame position
\begin{enumerate}
    \item Check that the Go position is elementary and even.
    \item Break up the Game into independent subpositions 
    (using groups of stones that are \textit{alive}, or whose life 
    does no depend on other results).
    \item Compute the value of the chilled game for each subposition.
    \item Sum the total to determine the outcome of chilled game. The outcome 
    of the unchilled game will be the same.
\end{enumerate}
Following \cite{berlekamp1994mathematical}, we add one extra twist to the chilled game
- this just corresponds to the \textit{temperature}
These are simply introduced for computational convenience - one could achieve the same 
results with e.g. $* = \{-1,1\}$

Then, to find \textbf{a} possible winning move for the winning player, examine 
the new sum resulting from a move on each of the subpositions. If the sum is 
still positive (resp. negative), that move is a winning move for Black 
(resp. White) - since this means that that move leads to a game with the same outcome!
A move that brings the total to 0 is also winning, provided that 
player is the first player. 

\subsubsection{Corridors}
The theory can be applied to give some precise values to subpositions in the chilled version of the game - 
in addition to the subpositions for which we already computed infinitesimal values in \S \ref{infinitesimals}.
\begin{theorem}
    \label{blocked_corridor}
    In the chilled game, a ``blocked corridor'' of length $n$ with $n-2$ markings is worth $2^{1-n}$ ($n\geq1$).
    (up to sign, depending on whether surrounding stones are Black or White).
\end{theorem}

\begin{figure}[H]
    \centering
    \begin{psgopartialboard*}{(1,1)(4,3)}
        \stone{black}{a}{1}
        \stone{black}{b}{1}
        \stone{black}{c}{1}
        \stone{black}{d}{1}
        \stone{black}{d}{2}
        \stone{black}{d}{3}
        \stone{black}{c}{3}
        \stone{black}{b}{3}
        \stone[\marklb{$\tau$}]{black}{a}{3}
        \stone[\marklb{$\tau$}]{white}{a}{2}
        
    \end{psgopartialboard*}
    \caption{Chilled blocked corridor of length 2}
    \label{fig:my_label}
\end{figure}
\begin{proof}(\cite{berlekamp1991introductory}[\S 4.3]) 
We proceed by induction. For $n=1$, notice that $-1 + \int 1 = *$. \\
Now for $n \geq 2$, a move by either player adjusts the markings on the diagram by 1

\end{proof}

\begin{theorem}
    \label{unblocked_corridor}
    In the chilled game, an ``unblocked corridor'' of length $n$ with $n-4$ markings is worth $2^{3-n}$ ($n\geq2$)
    (up to sign, depending on whether surrounding stones are Black or White).
\end{theorem}

\begin{figure}[H]
    \centering
    \begin{psgopartialboard*}{(1,1)(5,3)}
        \stone[\marklb{$\tau$}]{black}{a}{1}
        \stone{black}{b}{1}
        \stone{black}{c}{1}
        \stone{black}{d}{1}
        \stone{black}{e}{1}
        \stone{black}{d}{3}
        \stone{black}{c}{3}
        \stone{black}{b}{3}
        \stone[\marklb{$\tau$}]{black}{a}{3}
        \stone{black}{e}{3}
        
        \markpos{\marksq}{c}{2}
        \stone[\marklb{$\tau$}]{white}{e}{2}
        \stone[\marklb{$\tau$}]{white}{a}{2}
    \end{psgopartialboard*}
    \caption{Chilled unblocked corridor of length 3}
\end{figure}
The proof of this second result is similar by induction.

\subsubsection{Solving A Full Endgame Position}
We are now in measure to fully solve some endgame positions. The following 
example from an endgame position on a $9\times9$ board 
is adapted from \cite[\S 2]{berlekamp1994mathematical}

\begin{figure}[H]
    \centering
    \begin{psgoboard*}[9]
    \stone{black}{d}{9}
    \stone{black}{g}{9}
    \stone{white}{h}{9}
    
    \stone{black}{b}{8}
    \stone{black}{c}{8}
    \stone{black}{d}{8}
    \stone{white}{e}{8}
    \stone{white}{f}{8}
    \stone{white}{g}{8}
    \stone{white}{j}{8}
    
    \stone{black}{b}{7}
    \stone{black}{f}{7}
    \stone{white}{a}{7}
    \stone{white}{e}{7}
    \stone{white}{g}{7}
    \stone{white}{h}{7}
    
    \stone{black}{a}{6}
    \stone{black}{b}{6}
    \stone{black}{c}{6}
    \stone{black}{d}{6}
    \stone{black}{e}{6}
    \stone{white}{g}{6}
    
    \stone{black}{e}{5}
    \stone{black}{f}{5}
    \stone{white}{b}{5}
    \stone{white}{g}{5}
    \stone{white}{h}{5}
    
    \stone{black}{f}{4}
    \stone{black}{h}{4}
    \stone{black}{j}{4}
    \stone{white}{b}{4}
    \stone{white}{c}{4}
    \stone{white}{d}{4}
    \stone{white}{e}{4}
    \stone{white}{g}{4}
    
    \stone{black}{f}{3}
    \stone{black}{h}{3}
    \stone{white}{b}{3}
    
    \stone{black}{e}{2}
    \stone{black}{f}{2}
    \stone{black}{h}{2}
    \stone{black}{j}{2}
    \stone{white}{a}{2}
    \stone{white}{b}{2}
    \stone{white}{c}{2}
    \stone{white}{d}{2}
    
    \stone{black}{c}{1}
    \stone{black}{e}{1}
    \stone{black}{h}{1}
    \stone{white}{b}{1}
    \stone{white}{f}{1}
    
    \markpos{\marklb{A}}{e}{9}
    \markpos{\marklb{B}}{d}{7}
    \markpos{\marklb{C}}{f}{6}
    \markpos{\marklb{D}}{a}{5}
    \markpos{\marklb{E}}{j}{5}
    \markpos{\marklb{F}}{e}{3}
    \markpos{\marklb{G}}{g}{3}
    \markpos{\marklb{H}}{d}{1}
    
    \end{psgoboard*}
    \caption*{White to move and win.}
\end{figure}
Fist, let us find alive groups to check that positions $A,B,\ldots, H$
are independent.
\begin{figure}[H]
    \centering
    \begin{psgoboard}[9]
    \stone{black}{d}{9}
    \stone{black}{g}{9}
    \stone{white}{h}{9}
    
    \stone{black}{b}{8}
    \stone[\marklb{$\tau$}]{black}{c}{8}
    \stone[\marklb{1}]{black}{d}{8}
    \stone{white}{e}{8}
    \stone{white}{f}{8}
    \stone{white}{g}{8}
    \stone{white}{j}{8}
    
    \stone{black}{b}{7}
    \stone{black}{f}{7}
    \stone{white}{a}{7}
    \stone{white}{e}{7}
    \stone[\marklb{2}]{white}{g}{7}
    \stone[\marklb{$\tau$}]{white}{h}{7}
    
    \stone{black}{a}{6}
    \stone{black}{b}{6}
    \stone{black}{c}{6}
    \stone{black}{d}{6}
    \stone{black}{e}{6}
    \stone{white}{g}{6}
    
    \stone{black}{e}{5}
    \stone{black}{f}{5}
    \stone{white}{b}{5}
    \stone{white}{g}{5}
    \stone{white}{h}{5}
    
    \stone{black}{f}{4}
    \stone{black}{h}{4}
    \stone{black}{j}{4}
    \stone[\marklb{$\tau$}]{white}{b}{4}
    \stone[\marklb{3}]{white}{c}{4}
    \stone{white}{d}{4}
    \stone{white}{e}{4}
    \stone{white}{g}{4}
    
    \stone{black}{f}{3}
    \stone{black}{h}{3}
    \stone{white}{b}{3}
    
    \stone{black}{e}{2}
    \stone{black}{f}{2}
    \stone[\marklb{$\tau$}]{black}{h}{2}
    \stone[\marklb{4}]{black}{j}{2}
    \stone{white}{a}{2}
    \stone{white}{b}{2}
    \stone{white}{c}{2}
    \stone{white}{d}{2}
    
    \stone{black}{c}{1}
    \stone{black}{e}{1}
    \stone{black}{h}{1}
    \stone{white}{b}{1}
    \stone{white}{f}{1}
    
    \markpos{\marklb{A}}{e}{9}
    \markpos{\marklb{B}}{d}{7}
    \markpos{\marklb{C}}{f}{6}
    \markpos{\marklb{D}}{a}{5}
    \markpos{\marklb{E}}{j}{5}
    \markpos{\marklb{F}}{e}{3}
    \markpos{\marklb{G}}{g}{3}
    \markpos{\marklb{H}}{d}{1}
    
    \end{psgoboard}
\end{figure}
First, note that the Black can guarantee two eyes for group (1) by playing, e.g. at A8, A9 or B9, and 
so group 1 is alive. \\
Next, White can guarantee life for group (2) (and other stones in the upper right corner) even if 
black attacks at J5, with a response at J6 that makes two eyes. \\
Similarly, White's group (3) can get two eyes with plays at A5, A4, D3, E3 or D1. \\
Finally, Black's group (4) already has two eyes. \\
Therefore, positions involving only these groups (and potentially isolated stones that cannot connect 
to any group other than one that is already alive), are independent from one another. Therefore, we 
can analyze the games formed by positions $A, B. \ldots H$ separately and sum to get the result 
for the full game. \\

Now, we can analyze the values of these sub-positions individually in the chilled game. These are immediate using 
existing results:
\begin{itemize}
    \item $A = \downarrow$: this is a white corridor of length 2 with a black stone at the end 
    (\ref{basic_infinitesimals})
    \item $B = 1/2$: by Theorem \ref{blocked_corridor}, since this is a Black blocked corridor of length 2 invaded by White
    \item $C = *$: this is a black stone surrounded by White which can be captured (resp saved) in one 
    White (resp. Black) move (\ref{basic_infinitesimals})
    \item $D = -1/4$: by Theorem \ref{blocked_corridor}, since this is a White blocked corridor of length 3 invaded by White
    \item $F = -1/4$: this position is analogous to the corridor in $D$ 
    \item $G = \Uparrow *$: this is a blocked Black corridor of length 3 with a White stone at the end 
    \ref{derived_infinitesimals}
    \item $H = *$:  this is a black stone surrounded by white which can be captured (resp saved) in one 
    White (resp. Black) move (\ref{basic_infinitesimals})
\end{itemize}
$E$ requires a bit more work. On one hand, White's move a J5 dominates all of their other options, but 
adds a marking, for a net result of 0. 
On the other hand, Black's move at J5 removes a White marking, and elicits a White response at 
J6, which adds a White marking. Black surround no territory, while White surrounds two intersections 
but has two markings. So this position is in fact equal to $\downarrow = \{*|0\}$.  \\

Therefore, the outcome of this position is given by 
\[
    \underbrace{\downarrow + \downarrow + \Uparrow*}_{*} + \underbrace{1/2 - 1/4 - 1/4}_{0} +\underbrace{* + *}_{0} = *
\]
Since White plays first, they can enforce a win. Now, to determine the winning move for White:
\begin{itemize}
    \item A move on $A$ at E9 changes the value from $\uparrow$ to $*$, so the sum becomes $\uparrow$, which 
    is a win for Black
    \item A move on $B$ at D2 changes the value from $-1/4$ to $0$, so the sum becomes $1/4$, which is a win for Black
    \item A move on $C$ at F6 changes the value from $*$ to $0$, so the sum becomes $0$, which is a win for White 
    (since they will play after Black)
    \item A move on $D$ at A5 changes the value from $-1/4$ to $0$, so the sum becomes $1/4$, which is a win for 
    Black
    \item A move on $E$ at J5 changes the value from $\downarrow$ to $0$, so the sum becomes $\uparrow *$, which 
    is a win for Black
    \item A move on $F$ at E3 changes the value from $-1/4$ to $0$, so the sum becomes $1/4$, which is a win 
    for Black
    \item A move on $G$ at G3 changes the value from $\Uparrow*$ to $\uparrow$, so the sum becomes $\downarrow$,
    which is a win for White
    \item A move at $H$ on D1 changes the value from $*$ to $0$, so the sum becomes $0$, which is a win for 
    White (since they will play after Black)
\end{itemize}
Therefore, winning moves for White are D1, G3 and F6. Any other move loses to Black 
(the position with value $\downarrow$ is best since $\downarrow < 0$). \\
A similar analysis can then be applied to the resulting position depending on Black's response.
To see why White wins by one point in the unchilled game, notice that if we warm the sum for the position,
\[
    \int * = \left\{ 1 + \int \{ | \} \Bigg| -1 + \int \{ | \} \right\} = \{ 1 | -1 \}
\]
So if White plays first, they get to move to $-1$, which is a win by 1 point.

Note that we did not make use of Theorem \ref{bypassing} in the above example, 
but it is frequently used to simplify a position and recover one in which we can 
apply theorems \ref{blocked_corridor} or \ref{unblocked_corridor} or one of the infinitesimals 
we defined. \\

Some more complicated positions can be analyzed using these tools, though
they require proving several more theorems on chilled positions, one important 
example being multiple corridor invasions by one connected group. 
See \cite[\S 4]{berlekamp1994mathematical} for a more exhaustive treatment of
positions that can be solved using the theory.
\end{document}