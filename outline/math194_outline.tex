\documentclass{article}[12pt]

\usepackage{amsmath}

\title{
    Playing Optimal Go Endgames with Combinatorial Game Theory \\
    \large
}
\date{}
\author{Anders Poirel | apoirel@ucsc.edu}

\begin{document}
    
\maketitle

\begin{abstract}
    Combinatorial game theory studies deterministic, perfect-information 
    games with guaranteed termination. As a perfect-information, symmetric 
    two-player game, Go is one of the canonical objects of study in the field.
    While the combinatorial complexity of the game is such that it cannot be fully solved,
    Go endgames can be decomposed into simpler objects where combinatorial game theory can 
    be used to achieve optimal play. With the small victory margins observed at high
    levels of (human) play, using results from combinatorial game theory can have a 
    noticeable impact on one's odds of winning 
\end{abstract}

\end{document}
