\documentclass[../math_194_paper.tex]{subfiles}

\begin{document}

\section{Combinatorial Game Theory for Go}

This sections goes over a few fundamental concepts of combinatorial game 
theory, illustrating each with examples of Go. Concepts that do not apply 
to the analysis of go endgames (e.g. cardinal numbers) are glossed over.

\subsection{Basic Concepts}
Review of Conway's definition of Game. Fundamental structures and 
concepts

\subsubsection{Mathematical Re-formulation of Rules of Go}
Detail on how this definition is a natural fit for the Game of go. 
Re-formulation of rules of game for more convenient analysis 
(while showing their quasi-equivalence with AGA rules). 

\subsection{Conway Induction}
Fundamental theorems. Conway induction as the main tool of proof in the 
theory

\subsection{Arithmetic of Games}
Games as an algebraic group/ring/field. Illustration of the additive 
structure of the game of Go

\subsection{Simplifying Games}

Some useful theorems that can be used to simplify positions 

\subsection{Surreal Numbers}
The next two subsections explore some sub-classes of Conway Numbers 
(also known as surreal numbers) and how they can be used to evaluate 
certain moves in Go.

\subsubsection{Integers and Fractions}

Integers as finished game in normal rules. Fractional point value 
of endgame plays.

\subsubsection{Inifinitesimals}

Infinitesimals as moves with smaller than fractional value. 
Preview of the concept of chilling.

\end{document}