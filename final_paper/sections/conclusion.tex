\documentclass[../math194_paper.tex]{subfiles}

\begin{document}

It is conceivable that one could solve the  in \S \ref{endgame_position} by carefully analyzing 
the game tree. However, the techniques presented give a much more systematic - 
and faster, once one is familiar with the tools - way of carrying out this analysis,
which carries a definite advantage as positions become larger.
In fact, without these tools, larger version of similar problems can stump 
even the best human professionals or general-purpose AI techniques.
Furthermore, this analysis can be useful before the endgame - 
professional players frequently make midgame moves based on small endgame subtleties 
\cite[\S 5.1]{berlekamp1994mathematical}. \\

However, applying these technique to other types of Go problems meets considerable
difficulties - for instance, in many positions more than a point is in play, 
and while warming inverts chilling by one point, chilling by two or more points is not 
inverted by warming as many times \cite[\S 5.2]{berlekamp1994mathematical}.
Thus, extending the results of combinatorial game theory to solve more categories of Go positions 
remains an area of active research. 

\end{document}