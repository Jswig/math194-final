\documentclass[../math194_paper.tex]{subfiles}

\begin{document}

\section{Combinatorial Game Theory for Go}

This sections goes over a few fundamental concepts of combinatorial game 
theory, illustrating each with examples of Go. Concepts that do not apply 
to the analysis of go endgames (e.g. cardinal numbers) are glossed over.
We follow broadly the approach of \cite{schleicher2006introduction}, starting 
with \textit{games} as the fundamental structure and only introducing 
\textit{numbers} later.

\subsection{Basic Concepts}

\begin{definition}[Game] 
    A \textit{game} is defined recursively as follows:
    \begin{itemize}
        \item[(1)] Let $L$, $R$ be two sets of game. Then ordered pair $G = (L, R)$ is a 
        game  
        \item[(2)] (Descending Game Condition) There is no infinite sequence of games  
        $G^i = (L^i, R^i)$ with $G^{i+1} \in L^i \cup R^i$ for all $i \in \N$
    \end{itemize}
\end{definition}
The elements of $L$ and $R$ in the above are called the \textit{left} a and \textit{right options}
of $G$ respectively. \\
Many sources such as \cite{berlekamp1994mathematical} write $\{L \mid R\}$ for the ordered pair 
$(L,R)$. 

For instance consider the following Go position. 

\subsection{Conway Induction}

The following theorem presents gives a form of induction on games. 
This has applications to almost every result in the theory, since most 
proofs rely on being able to manipulate the recursive structure of games.

\begin{theorem}[Conway Induction]
\label{induction}
For $n \geq 1$ let $P(G_1, \ldots, G_n)$ be a $n$-ary relation on games.
For all $i \in \{1, \ldots n\}$ suppose $P(G_1, \ldots, G_i', \ldots, G_n)$ implies 
$P(G_1, \ldots, G_i, \ldots, G_n)$. Then $P(G_1, \ldots, G_n)$ holds for any $n$-tuple 
of games.
\end{theorem}

\begin{proof}
    Suppose $P(G_1, \ldots, G_n)$ is false. 
    (In the general case, making this selection requires the Axiom of Choice,
    since a game can have infinitely many options).
    Thus, $\{G_k\}_k$ is an inifinite sequences of games, where $G_{k+1} \in L_k \cup R_k$
    for all $k$. This is a contradiction, since by the Descending Game Condition no 
    such infinite sequence exists. 
\end{proof}

It can be shown that Conway Induction implies the Descending Game Condition \cite{schleicher2006introduction}
As such, they are equivalent.

\subsection{Winning a Game}

\subsubsection{Mathematical Re-formulation of Rules of Go}
Detail on how this definition is a natural fit for the Game of go. 
Re-formulation of rules of game for more convenient analysis.

The concept of \textit{numbers} will make this correspondence even simpler.

\subsection{Arithmetic of Games}
Games as an algebraic group/ring/field. Illustration of the additive 
structure of the game of Go

Other operations besides addition and negation can be defined on some 
classes of games (in particular, the class of game introduced in the next section, 
\textbf{numbers}, forms a field \cite{schleicher2006introduction}),
 but these two operations are the only ones we require in this study of Go endgames. 

\subsection{Some Useful Surreal Numbers}

\subsubsection{Integers} 

\subsubsection{Relating the Mathematical Game to the Real Game}

Equipped with the notion of integers, it is possible to clarify the relation 
between the ``mathematical'' rules of Go presented above and the standard AGA 
rules.

\subsubsection{Fractions}

The applications of fractions to Go will become clear once we introduce 
the chilling and warming operators to study endgame positions. 

Arbitrary real numbers can then be constructed as such using fractions.
TODO: show example. 
However, these do not show up in the endgame analysis 
techniques discussed here.

\subsubsection{Inifinitesimals}

Infinitesimals as moves with smaller than fractional value. 

John H. Conway's theory of surreal numbers also includes \textbf{ordinal numbers} 
\cite{schleicher2006introduction}. However, while these do not show either in the analysis 
of Go endgames, it is interesting to note that surreal numbers provide a unified 
framework for treating infinitesimals, real numbers and ordinal numbers, and that 
Go endgame was the first motivator for the study of this class 
(cf. \cite[Prologue]{conway2000numbers}).

\subsection{Simplifying Games}

This section presents several theorems for symplifying games that will be 
useful in the proofs of several important results as well as in computing 
values of endgames positions. 

\begin{theorem}[Number Avoidance Theorem]
    
\end{theorem}

\begin{theorem}[Deleting Dominated Options]
    Let $G$ be a game with $G^L \in \mathcal{G}^L$ and $G^R \in \mathcal{G}^R$.
    Then value of $G$ is unchanged if options of $\mathcal{G}^L$ dominated by $G^L$
    are removed. Similarly, the value of $G$ is unchanged if options of $\mathcal{G}^R$ 
    which are dominated by $G^R$ are removed. 
\end{theorem}

\begin{proof}
    A similar argument holds for removing dominated right options of $G$ \qed
\end{proof}

\begin{theorem}[Bypassing Reversible Options]
\end{theorem}

\begin{proof}

\end{proof}

\end{document}