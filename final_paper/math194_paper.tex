\documentclass{article}

\usepackage{mathpb} % custom mathematics package including AMSmaths/fonts

\usepackage{multicol}
\usepackage{psgo}
\usepackage{amsrefs}
\usepackage{subcaption}
\usepackage{hyperref}
\usepackage{graphicx}
\usepackage{float}
\usepackage{subfiles}

\theoremstyle{plain}
\newtheorem{theorem}{Theorem}[section]
\newtheorem{corollary}{Corollary}[theorem]
\newtheorem{lemma}{Lemma}[theorem]

\theoremstyle{definition}
\newtheorem{definition}{Definition}[section]

\hypersetup{linktoc=all}

\title{Solving One-Point Go Endgames with Combinatorial Game Theory}
\author{Anders Poirel - apoirel@ucsc.edu \\ University of California, Santa Cruz}
\date{Spring 2021}

\begin{document}

\maketitle

\begin{figure}[ht]
    \centering
    \includegraphics[width=\textwidth]{prince_genji.jpg}
\end{figure}

\begin{abstract}
    Combinatorial game theory studies deterministic, perfect-information 
    games with guaranteed termination. As a perfect-information, symmetric 
    two-player game, Go is one of the canonical objects of study in the field.
    While the combinatorial complexity of the game is such that most positions cannot 
    be solved solved, 
    many Go endgame positions can be decomposed into simpler objects which can fully
    analyzed using results from combinatorial game theory
\end{abstract}

\newpage

Illustration: Toyokuni III, \textit{Prince Genji Watching Two Ladies Playing Go }, 
Izumiya Ichibei, 1849-1851.

\tableofcontents
\newpage 
\section{Introduction}

Go is known for its combinatorial complexity - a lower bound $10^{10^48}$ has 
been found on the number of possible Go games on a 19$\times$19 board. While this complexity 
makes it impossible in practice to give exact results in most positions, it also makes 
the game almost 

by introducing illustrating the fundamental concepts of the theory immediately 
with Go positions, I hope to convey how the theory naturally emerges from the rules 
of Go, despite being applicable to a much wider variety of two-player deterministic
games. 

\subfile{sections/game}

\section{Combinatorial Game Theory for Go}

\subfile{sections/game_theory}

\subfile{sections/simplifying_games}

\subfile{sections/surreal_numbers}

\section{One-point Go Endgames}

\subfile{sections/endgame}

\subfile{sections/temperature_theory}

\subfile{sections/positions_analysis}

\section{Conclusion}

\subfile{sections/conclusion}

\bibliography{references}

\end{document}