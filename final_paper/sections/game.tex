\documentclass[../math194_paper.tex]{subfiles}

\begin{document}

\section{Game of Go}

\subsection{Background}
Go, also known as \textit{weiqi} (China) or \textit{baduk} (Korea) has been played 
since at least the 4th century BCE, with its first known appearance in the writings 
of Chinese scholars. At first an aristocratic pastime, Go has been the subject of 
professional study since the 17th century when the Tokuguwa dynasty in Japan established 
a national system of Go schools and tournaments \cite{agahistory}.\\

Go was popularized in the West in the early 20th century by the German engineer Oskar 
Korschelt, who introduced the game to many contemporary German mathematicians \cite{korschelt2012theory}.\\

Due to its complexity, and reputation as a benchmark for human intelligence,
Go has long been a major topic of research in mathematics and computer science: the first
artificial intelligence program for Go was written in 1968, and Deepmind's AlphaGo victory 
over the top professional human player in 2016 was seen as a major benchmark in the field 
of artificial intelligence \cite{silver2017mastering}.
    
\subsection{Rules of Go}

The following is adapted from the official American Go Association (AGA) rules
\cite{agarules}. While in almost all case equivalent to the more common Chinese 
or Japanese rules, some scoring ambiguities in the latter are difficult to mathematize,
resulting in a more cumbersome theory \cite[Appendix. 1]{berlekamp1994mathematical}. 

\subsubsection{AGA Rules}
\begin{enumerate}
\item (The Board) Go is played between two players on grid, often $19\times19$. The board
is initially empty. 
\item (Play) Black and White alternate in moving, starting with Black. A move consists
either of placing a \textit{stone} of one's color on an legal empty intersection of the board, or passing.
(see rules 3, 4 and 5).
\item (Capture) 
\begin{enumerate}
    \item  A \textit{liberty} of a stone is a vacant, adjacent intersection (horizontally or vertically). 
    \item Stones of the same color are \textit{connected} if they are adjacent (horizontally or vertically). 
    \item Stone are part of the same \textit{group} if there are there exists a path between 
     connected stones of the same color between them. Stones in the same group share liberties.
    \item After a player moves, any stone or group of stones which is completely surrounded 
    by stones of the other color, leaving no liberties, is captured and removed from the 
    board \footnote{Go gets its Chinese name from this game mechanic, \textit{weiqi} means 
    ``surrounding game''}. 
    Such stones become \textit{prisoners} of the capturing player. 
    It is  illegal for a
    player to move such as to create a group of their own stones that does not have any
    liberties after any surrounded group of the opposed color are captured.
\end{enumerate}
\item (Repeated Board Positions / \textit{Ko}) it is illegal to play 
a move that recreates a previous full position from the game, with the 
same player to move
\item (Passing) On their turn, a player may as their move pass by handing their opponent a stone as a prisoner,
instead of placing a stone on the board
\item (Ending the Game) 
\begin{enumerate} 
    \item The game ends when Black passes and White's subsequent move is also a pass.
    \item Any stones which both players agree could not escape capture if the game continued, are removed 
    from the board and treated as prisoners of the player who could capture them \footnote{In practice, if
    there is a disagreement, the players
    can continue playing until said stones are captured. It can be shown that this results in the same score
    if the stone are    
    captured using the minimal number of moves. This rule exists to speed up the game such that
    two skilled players can avoid playing out trivial positions where the result is known ahead of time. 
    As such, we can ignore this part of the rule in our mathematical treatment of the rules.}.
\end{enumerate}
\item (Counting Score) At the end of the game, the score is determined as follows:
\begin{enumerate} 
    \item Empty intersections on the board which are entirely surrounded by stones of a single color 
    are considered \textit{territory} of the player of that color. An empty point is surrounded by stones of a
    single color if there is no path to a stone of the opposing color from that point by moving only to adjacent
    empty points.
    \item (\textit{seki} and \textit{dame}) In some cases, intersections are not completely surrounded by a
    single color, in which case they do no count as anyone's territory.
    \item A player's score is given by all the intersection in their territory, less prisoners of their color 
    held by the opponent. White adds 7.5 points (a \textit{komi}) to their score in compensation for moving 
    second \footnote{The half-point is added to ensure the game never ends in a draw.}.
\end{enumerate}
\end{enumerate}

\subsection{Rule Examples}
The following illustrates the rules with some partial board positions. In this (and all 
subsequent figures) a sequence of moves will be numbered (or lettered) in the appropriate 
order, i.e. move A is played before move B.

\begin{figure}[H]
\begin{subfigure}[b]{0.45\linewidth}
\centering
\begin{psgopartialboard*}{(1,1)(3,4)}
    \stone{black}{a}{2}
    \stone{black}{b}{2}
    \stone{black}{b}{3}
    
    \markpos{\marklb{1}}{a}{1}
    \markpos{\marklb{2}}{b}{1}
    \markpos{\marklb{3}}{c}{2}
    \markpos{\marklb{4}}{c}{3}
    \markpos{\marklb{5}}{b}{4}
    \markpos{\marklb{6}}{a}{3}
\end{psgopartialboard*}
\caption{The black stones form a group which has 6 liberties}
\end{subfigure}
\quad
\begin{subfigure}[b]{0.45\linewidth}
\centering
\begin{psgopartialboard*}{(1,1)(3,4)}
    \stone{black}{a}{3}
    \stone{black}{b}{3}
    \stone{black}{b}{1}
    \stone{white}{a}{2}
    \stone{white}{a}{1}
    
    \move{b}{2}
\end{psgopartialboard*}
\caption{Black's move at 0 captures the two white stones}
\end{subfigure}

\begin{subfigure}[b]{0.45\linewidth}
\centering
\begin{psgopartialboard*}{(1,1)(3,4)}
    \stone{white}{a}{2}
    \stone{white}{b}{2}
    \stone{white}{b}{1}
    
    \markpos{\marklb{a}}{1}{1}
\end{psgopartialboard*}
\caption{Black cannot legally move at \textit{a} since it would immediately be captured by White}
\end{subfigure}
\quad 
\begin{subfigure}[b]{0.45\linewidth}
\centering
\begin{psgopartialboard*}{(1,1)(3,4)}
    \stone{white}{a}{2}
    \stone{white}{b}{2}
    \stone{white}{b}{1}
    \stone{black}{a}{3}
    \stone{black}{b}{3}
    \stone{black}{c}{2}
    \stone{black}{c}{1}
    \stone[\marklb{A}]{black}{a}{1}
\end{psgopartialboard*}
\caption{Black's move at A is legal and captures the white group}
\end{subfigure}

\begin{subfigure}[b]{0.45\linewidth}
\centering
\begin{psgopartialboard*}{(1,1)(4,4)}
    \stone{white}{b}{3}
    \stone{white}{a}{2}
    \stone{white}{b}{1}
    \stone{black}{c}{3}
    \stone{black}{d}{2}
    \stone{black}{c}{1} 
    \stone[\marklb{A}]{white}{c}{2}
    \stone[\marklb{B}]{black}{b}{2}
\end{psgopartialboard*}
\caption{
    \textit{Ko}: after Black responds to white move's at A by capturing the stone with a move at B, 
    White cannot recapture by moving at again at A, unless they play somewhere else on the board first
}
\end{subfigure}
\quad
\begin{subfigure}[b]{0.45\linewidth}
\centering
\begin{psgopartialboard*}{(1,1)(4,4)}
    \stone{white}{a}{3}
    \stone{white}{c}{2}
    \stone{white}{b}{3}
    \stone{white}{b}{2}
    \stone{white}{a}{1}
    \stone{white}{d}{2}
    \stone{white}{d}{1}
    \markpos{\markma}{a}{2}
    \markpos{\markma}{b}{1}
    \markpos{\markma}{c}{1}
    
    \stone{black}{b}{4}
    \stone{black}{c}{4}
    \stone{black}{d}{4}
    \stone{black}{c}{3}
\end{psgopartialboard*}
\caption{
    White has 3 points of territory, marked X. The other intersections are not territory for either player.
}
\end{subfigure}
\end{figure}

\begin{figure}[H]
\centering
\begin{psgopartialboard*}{(1,1)(5,5)}
    \stone{black}{a}{1}
    \stone{black}{a}{2}
    \stone{black}{a}{3}
    \stone{black}{b}{3}
    
    \stone{black}{d}{1}
    \stone{black}{d}{2}
    \stone{black}{d}{3}
    \stone{black}{d}{4}
    \stone{black}{c}{4}
    
    \stone{white}{a}{4}
    \stone{white}{b}{4}
    \stone{white}{c}{1}
    \stone{white}{c}{2}
    \stone{white}{c}{3}
    
    \markpos{\marklb{a}}{b}{1}
    \markpos{\marklb{b}}{b}{2}
\end{psgopartialboard*}
\caption*{
    (d) \textit{Seki} (mutual life): If Black plays at \textit{a} White can capture by playing at \textit{b} 
    and vice versa. Thus, neither the black nor the white group are \textit{dead}, so the intersections 
    \textit{a,b} are not territory for either player. 
}
\end{figure}

Finally, we highlight a fundamental pattern of the game: \textit{two eyes}. 
Eyes are empty intersections that are surrounded in all
four directions by stones of the same color. A group with two eyes is uncapturable, 
since an opponent's move on either eye would be immediately captured, given that the group
has an ``extra'' liberty (see Rule 3). Such groups are said to be \textit{alive}. 

\begin{figure}[H]
\centering
\begin{psgopartialboard*}{(1,1)(5,5)}
    \stone{white}{a}{1}
    \stone{white}{b}{2}
    \stone{white}{a}{3}
    \stone{white}{b}{3}
    \stone{white}{c}{2}
    \stone{white}{c}{1}
    \markpos{\marklb{a}}{a}{2}
    \markpos{\marklb{b}}{b}{1}
    \stone{black}{a}{4}
    \stone{black}{b}{4}
    \stone{black}{c}{4}
    \stone{black}{c}{3}
    \stone{black}{d}{3}
    \stone{black}{d}{2}
    \stone{black}{d}{1}
\end{psgopartialboard*}
    \caption*{The White group is alive and  has two eyes at \textit{a} and \textit{b}. 
    Black cannot capture it.}
\end{figure}

In practice, competent players will rarely flesh out the full two eyes, instead
ensuring that they have the possibility of making two eyes if the position is played out. 
As a convention, we will mark alive groups on diagrams with the symbol $\tau$ (e.g. when 
the eyes of the group are outside of the diagram). \\

Alive groups are in a sense independent from the rest of the board - no move from the 
opponent can affect the status of such a group. This fact will be of crucial important 
in analyzing endgames, where we will be able to argue that sub-positions are independent 
based on groups being alive.

\end{document}