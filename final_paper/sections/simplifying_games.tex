\documentclass[../math194_paper.tex]{subfiles}

\begin{document}
\subsection{Simplifying Games}

This section presents several theorems for simplifying games that are 
used in the proofs of several important results as well as in computing 
values of endgames positions. 

The following result says that we can ignore moves that are worse than 
others (for the moving player) in computing the value of a game, since 
an optimal player will always ignore them.

\begin{definition}[Dominated Option] Let $G$ be a game.
    \begin{itemize}
        \item Left option $G_L$ is \textit{dominated} by another left option $G_L'$:
        $G_L \leq G_L'$.
        \item Right option $G_R$ is \textit{dominated} by another right option $G_R'$:
        $G_R \geq G_R'$.
    \end{itemize}
\end{definition}

\begin{theorem}[Deleting Dominated Options]
    Let $G$ be a game with $G^L \in \mathcal{G}^L$ and $G^R \in \mathcal{G}^R$ Then,
    \begin{itemize}
        \item The value of $G$ is unchanged if options of $\mathcal{G}^L$ dominated by $G^L$.
        \item The value of $G$ is unchanged if options of $\mathcal{G}^R$ 
    which are dominated by $G^R$ are removed. 
    \end{itemize}
\end{theorem}

\begin{figure}[H]

\end{figure}

This second result tells us that if the opponent can counter a move with a 
response that leads to a better positions (for them) than the original, then we 
should expect to play on the position resulting from that response.
\begin{definition}[Reversible Option] 
    Let $G$ be game. 
    \begin{itemize}
        \item Left option $G_L$ is \textit{reversible} through $G_{LR}$ if
        $G_L$ has right option $G_{LR} \leq G$. 
        \item Right option $G_R$ is \textit{reversible} through $G_{RL}$ if 
        $G_R$ has left option $G_{RL} \geq G$.
    \end{itemize}
\end{definition}

\begin{theorem}[Bypassing Reversible Options] \label{bypassing}\:
    \begin{itemize} 
        \item If $G$ has a left option $H$ that is reversible through $K=H_R$ then 
        replacing $H$ by all the left options of $K$ does not change the value of the game.
        \item If $G$ has a right option $H$ that is reversible through $K=H_L$ then 
        replacing $H$ with all the right options of $K$ does not change the value of the 
        game
    \end{itemize}
\end{theorem}

\begin{definition}[Short Game]
    A game is \textit{short} if it has finitely many positions.
\end{definition}

\begin{lemma}
    Go is a finite game
\end{lemma}
We can verify this by determining the following 
upper bound on the number of possible positions. On an $n \times n$ grid, each intersection
either has a White stone, a Black stone or is empty. Thus there are at most $(n^2)^3$
possible board positions. \footnote{
    The argument gets a lot more complicated if we account for prisoners directly. However 
    we can appeal to the Chinese rules - which we can prove yield the same result as 
    AGA or ``mathematical'' rules, but don't use prisoners - and substitute AGA games 
    with equal Chinese games in the argument.
}

\begin{definition}[Canonical Forms]
    \ref{canonical}
    In each equivalence class of finite games (under =), there is a unique game that has no dominated
    or reversible options. \\
    This game is called a \textit{canonical form}.
\end{definition}
This allows us to greatly simplify our analysis of Go positions by arguing on the canonical 
form directly. 

\end{document}