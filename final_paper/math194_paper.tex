\documentclass{article}

\usepackage{mathpb} % custom mathematics package including AMSmaths/fonts

\usepackage{amsrefs}
\usepackage{tikz}
\usepackage{hyperref}
\usepackage{subfiles}

\theoremstyle{plain}
\newtheorem{theorem}{Theorem}[section]
\newtheorem{corollary}{Corollary}[theorem]
\newtheorem{lemma}{Lemma}[theorem]

\theoremstyle{definition}
\newtheorem{definition}{Definition}[section]

\hypersetup{linktoc=all}

\title{Playing Optimal Go Endgames with Combinatorial Game Theory}
\author{Anders Poirel - apoirel@ucsc.edu \\ University of California, Santa Cruz}
\date{Spring 2021}

\begin{document}

\maketitle

\begin{abstract}
    Combinatorial game theory studies deterministic, perfect-information 
    games with guaranteed termination. As a perfect-information, symmetric 
    two-player game, Go is one of the canonical objects of study in the field.
    While the combinatorial complexity of the game is such that it cannot be fully solved,
    Go endgames can be decomposed into simpler objects where combinatorial game theory can 
    be used to achieve optimal play. With the small victory margins observed at high
    levels of (human) play, using results from combinatorial game theory can have a 
    noticeable impact on one's odds of winning 
\end{abstract}

\newpage
\tableofcontents
\newpage 

\section{Introduction}

This seeks a balance 
by introducing illustrating the fundamental concepts of the theory immediately 
with Go positions. I hope to convey how the theory naturally emerges from the rules 
of Go, despite being applicable to a much wider variety of two-player deterministic
games. 

\subfile{sections/game}

\subfile{sections/game_theory}

\subfile{sections/endgame}

\subfile{sections/conclusion}

\bibliography{references}

\end{document}