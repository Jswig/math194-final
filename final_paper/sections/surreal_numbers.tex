\documentclass[../math194_paper.tex]{subfiles}

\begin{document}

\subsection{Surreal Numbers}

Games whose right options are greater than their left options, such that this property holds recursively
for all options, are called \textit{numbers}, or more often, \textit{surreal numbers}, following
Knuth's treatment of the topic \cite{knuth1974surreal}. Formally,
\begin{definition}[Number]
    A game $G = \{\mathcal{G}_L \mid \mathcal{G}_R \}$ is a \textit{number}
    if $G_L < G_R$, and $G_L, G_R$ are numbers for all $G_L \in \mathcal{G}_L, G_R \in \mathcal{G}_R$.
\end{definition}
Rather than discussing surreal numbers in full generality, we will limit ourselves 
to constructing the surreal numbers that appear in our analysis of Go endgames.

\subsubsection{Integers} 
\begin{definition}[Natural Numbers]. A \textit{natural number} is defined a one of
\begin{enumerate}
    \item  $0 := \{ \emptyset \mid \emptyset \}$
    \item  $n := \{ (n-1) \mid \emptyset \}$
\end{enumerate}
\end{definition}
It is easy to see that $n+1 > n$.
Using this fact, notice how by adding dominated options (which from the theorem does not affect the equality), 
$\{ n\mid \} = \{ n-1, n \mid \} = \ldots \{ 1, 2, \ldots, n-1, n\mid \}$ which
yields the standard Von Neumann construction of the natural numbers \cite[\S 4]{enderton1977elements}
\cite[\S 2]{schleicher2006introduction}. 
Thus, all of the usual properties of natural numbers apply to this construction.

\begin{definition}[Negative Integers] A \textit{negative integer} is obtained by taking the 
negative (as defined in \S 2.2) of a natural number. i.e. for $n$ a natural number,
\[ (-n) := - \{n-1 \mid \} = \{ \mid -(n-1)\} \]
\end{definition}
These negative integers behave as expected, for instance, $n + (-n) = 0$ by Theorem \ref{reflexequal}
and the usual arithmetic properties can be derived by applying negation and distributing over the
corresponding identities using Theorem \ref{additionprop}. \\

Positive integers $n$ are games in which Left has $n$ ``free'' moves and
always wins: a move by left on $3 = \{2 \mid \}$ yields the new game $2 = \{1 | 0\}$ and so on.
Similarly, negative integers $n$ are games in which Right has $n$ ``free'' moves and always 
wins (in Go, replace Left and Right by White and Black, and ``free'' by ''legal'').

\subsubsection{Integers and Mathematized Rules}

Equipped with the notion of integers, we can further clarify the relation 
between the ``mathematical'' rules of Go presented in \S and the standard AGA 
rules. For instance, consider the following position
\begin{figure}[H]
\centering
\begin{psgoboard*}[4]
    \stone{black}{b}{1}
    \stone{black}{b}{2}
    \stone{black}{b}{3}
    \stone{black}{b}{4}
    \stone[\marklb{$\tau$}]{black}{a}{3}
    \markpos{\marklb{A}}{a}{1}
    \markpos{\marklb{B}}{d}{1}
    \stone{white}{c}{1}
    \stone{white}{c}{2}
    \stone{white}{c}{3}
    \stone{white}{c}{4}
    \stone[\marklb{$\tau$}]{white}{d}{2}
\end{psgoboard*}
\end{figure}
First, the half of the board $A$ is equal to 3: it is a game where Black can make 
3 legal moves.  If White were to play anywhere in that era, Black capture in one move, 
which recovers an extra legal move (returning the prisoner), which yields a new 
game where Black has $3$ moves. 
Similarly, side $B$ is equal to -3. 

Therefore, a real-world game will usually end once both players recognize that all the
subpositions are equal to integers. The score for Black is then the sum of all the 
positive integer subpositions, and the score for White is the absolute value of the 
sum of all the negative integer subpositions. \\

Note an inexperienced player could play badly and lose such territories, but 
remember that results for the values and order of a game are results for optimal play!

Using these facts about numbers, combined with Theorems \ref{equivalence} and 
\ref{canonical}, we can treat use Black  (resp. White) points and positive (resp. negative) 
integers interchangeably when analyzing subpositions.

\subsubsection{Rationals}

\begin{definition}[Dyadic Rational] 
\label{dyadic}
Let $m$ be an odd natural number, and $k$ be a natural 
number. Define \footnote{This construction is slightly cumbersome; this is because 
here we did show how to define a field on surreal numbers with a division operator.
Such an approach would make for a lengthy digression from the main results relevant 
to Go}
\begin{itemize}
    \item $\frac{1}{2} := \{0 \mid 1\}$; \quad  $\frac{0}{2^k} := 0$ 
    \item for $m-1 \neq 0$, $\frac{m}{2^k} := \left\{ \frac{m-1}{2^k} \Bigg| \frac{m+1}{2^k} \right\}$
\end{itemize}
\end{definition}

It is easy to verify that these behave as expected. For instance, 
\[
    \frac{1}{2} + \frac{1}{2} = \{0|1\} + \{0|1\} = \{0||0|\} + \{0||0|\} = 
    \left\{ 0 + \{0||0|\} |  \{0|\} + \{0||0|\} \right\} = \{0|\} = 1
\]
Negative fractions are obtained in the same way as negative integers.
For more details, see \cite[\S 0, \S 1]{conway2000numbers}.
These dyadic rationals will be used to evaluate moves which change the value of a game of Go by less than 
a full point.

\subsubsection{Infinitesimals} \label{infinitesimals}

Infinitesimals are a special type of game. They will correspond in Go to positions 
that have smaller value than fractions, but are still different from zero games.  
\begin{definition}[Useful Infinitesimals]
\begin{itemize} \label{basic_infinitesimals} \:
    \item $* := \{0 | 0 \}$
    \item $\uparrow := \{0 | * \}$
    \item $\downarrow := - \uparrow$
\end{itemize}

\begin{definition} \label{derived_infinitesimals} \:
\begin{itemize}
    \item $n \uparrow := \underbrace{\uparrow + \ldots + \uparrow}_{n \text{ times}}$
    \item $\Uparrow := \uparrow + \uparrow$
    \item $x* := x + *$
\end{itemize}
\end{definition}
\end{definition}

Unlike integers, these new numbers do not usually appear in the standard Go game. Instead, they 
make their appearance when we play the \textit{chilled game}. In the chilled game, the moving 
player must give a stone as a prisoner to their opponent - that is, a move effectively costs a 
point. The motivation for the chilled game will be made clearer in \S 4.\\

To simplify notation of chilled games, we will introduce the idea of \textit{markings} following 
\cite[\S 4.1]{berlekamp1994mathematical}. A filled square indicates a Black prisoner for White, and 
a empty square a White prisoner for Black. When a player moves, they either add a marking of
their color to the board, or remove one of the opponent's color (these both reduce that player's
net advantage by 1 point). Positive numbers of markings will mean Black markings, while a negative
number of markings will mean White markings.\\

Consider the following position in a chilled game:
\begin{figure}[H]
    \centering
    \begin{psgopartialboard*}{(1,1)(4,3)}
        \stone{black}{a}{1}
        \stone{black}{b}{1}
        \stone{black}{c}{1}
        \stone{black}{d}{1}
        \stone{black}{d}{2}
        \stone{black}{d}{3}
        \stone{black}{c}{3}
        \stone{black}{b}{3}
        \markpos{\marklb{A}}{b}{2}

        \stone[\marklb{$\tau$}]{black}{a}{3}
        \stone[\marklb{$\tau$}]{white}{a}{2}
        \stone[\marksl]{white}{c}{2}
    \end{psgopartialboard*}
\end{figure}
The options of Black and White on this position are respectively:
\begin{figure}[H]
    \centering
    \begin{subfigure}[b]{0.45\linewidth}
    \centering
    \begin{psgopartialboard*}{(1,1)(4,3)}
        \stone{black}{a}{1}
        \stone{black}{b}{1}
        \stone{black}{c}{1}
        \stone{black}{d}{1}
        \stone{black}{d}{2}
        \stone{black}{d}{3}
        \stone{black}{c}{3}
        \stone{black}{b}{3}
        
        \stone[\marksl]{black}{b}{2}
        \stone[\marklb{$\tau$}]{black}{a}{3}
        \stone[\marklb{$\tau$}]{white}{a}{2}
        \stone[\marksl]{white}{c}{2}
    \end{psgopartialboard*}
    \caption{Black moves at A, capturing the white stone but adding a marking, which cancels out the gained 
    point. Both players have 0 points, so this position is 0.}
    \end{subfigure}
    \quad
    \begin{subfigure}[b]{0.45\linewidth}
    \centering
    \begin{psgopartialboard*}{(1,1)(4,3)}
        \stone{black}{a}{1}
        \stone{black}{b}{1}
        \stone{black}{c}{1}
        \stone{black}{d}{1}
        \stone{black}{d}{2}
        \stone{black}{d}{3}
        \stone{black}{c}{3}
        \stone{black}{b}{3}
        
        \stone{white}{b}{2}
        \stone[\marklb{$\tau$}]{black}{a}{3}
        \stone[\marklb{$\tau$}]{white}{a}{2}
        \stone{white}{c}{2}
    \end{psgopartialboard*}
    \caption{White moves at A, saving the stone which Black threatened to capture, but removes 
    a black marking in the process. Both players have 0 points, so this position is 0.}
    \end{subfigure}
\end{figure}
Both the Left/Black and Right/White options are a 0 game. Thus this position is equal to $\{0 | 0\} = *$. \\

\textbf{Note:} the positions shown above are not the only Go positions equal to * - they are only 
one representative of the equivalence class of positions equal to *. For instance, any position 
with a isolated White stone that can be captured by a group of \textit{alive} Black stones but which White 
can save in one move by connected is equal to *. 
Taking the negation of any such games (i.e. switching the colors) also yields a game in the same equivalence class. \\

Similarly, $\uparrow$ also appears as a Go position in the chilled game:
\begin{figure}[H]
    \centering
    \begin{psgopartialboard*}{(1,1)(5,3)}
        \stone[\marklb{$\tau$}]{black}{a}{1}
        \stone{black}{b}{1}
        \stone{black}{c}{1}
        \stone{black}{d}{1}
        \stone{black}{e}{1}
        \stone{black}{e}{2}
        \stone{black}{e}{3}
        \stone{black}{d}{3}
        \stone{black}{c}{3}
        \stone{black}{b}{3}
        \stone{black}{a}{3}
        
        \stone[\marklb{$\tau$}]{white}{a}{2}
        \stone[\marksl]{white}{d}{2}
        \markpos{\marksl}{c}{2}
     \end{psgopartialboard*}
\end{figure}
The options of Black and White in this position are respectively
\begin{figure}[H]
    \centering
    \begin{subfigure}[b]{0.45\linewidth}
    \centering
    \begin{psgopartialboard}{(1,1)(5,3)}
        \stone[\marklb{$\tau$}]{black}{a}{1}
        \stone{black}{b}{1}
        \stone{black}{c}{1}
        \stone{black}{d}{1}
        \stone{black}{e}{1}
        \stone{black}{e}{2}
        \stone{black}{e}{3}
        \stone{black}{d}{3}
        \stone{black}{c}{3}
        \stone{black}{b}{3}
        \stone{black}{a}{3}
        
        \stone[\marksl]{black}{b}{2}
        
        \stone[\marklb{$\tau$}]{white}{a}{2}
        \stone[\marksl]{white}{d}{2}
        \markpos{\marksl}{c}{2}
     \end{psgopartialboard}
    \caption{Black's move at B2 secures two points of territory and a capture, but adds a marking
    for a net of 0 points. This dominates the option of playing at C2, since it gives up a point
    for territory for a net result of -1 points.}
    \end{subfigure}
    \quad
    \begin{subfigure}[b]{0.45\linewidth}
    \centering
    \begin{psgopartialboard}{(1,1)(5,3)}
        \stone[\marklb{$\tau$}]{black}{a}{1}
        \stone{black}{b}{1}
        \stone{black}{c}{1}
        \stone{black}{d}{1}
        \stone{black}{e}{1}
        \stone{black}{e}{2}
        \stone{black}{e}{3}
        \stone{black}{d}{3}
        \stone{black}{c}{3}
        \stone{black}{b}{3}
        \stone{black}{a}{3}
        
        \stone[\marklb{$\tau$}]{white}{a}{2}
        \stone[\marksl]{white}{d}{2}
        \stone{white}{b}{2}
     \end{psgopartialboard}
    \caption{White's move a B2 yields a position equal to *. This dominates the option of playing 
    at C2, which allows Black to capture two piece with their next move at B2, for a net result 
    of 1 point (in Black's favor.)}
    \end{subfigure}
\end{figure}
Thus, this position is equal to $\{0 | *\} = \uparrow$. $\downarrow$ is the same position with the colors reversed.
Similarly, we can obtain $\Uparrow*$ by adding a empty intersection and and marking to the ``corridor'' represented 
above, $3\uparrow$ by adding another empty intersection and marking, and so on.

The following properties are useful in analyzing positions where infinitesimals appear.
\begin{theorem}[Properties of Common Infinitesimals] \:
    \begin{enumerate}
        \item $* \parallel 0$
        \item $*+*=0$, $*=-*$
        \item $\uparrow > 0$
        \item $\uparrow \parallel *$
        \item $\Uparrow > *$
        \item $n \uparrow > (n-1) \uparrow$
    \end{enumerate}
\end{theorem}
These can easily be verified using the definitions. These results imply, for instance, that 
a game with a net value of $5*$ yields a Go score of $4$ or $6$, depending on who plays first 
(in effect, * is ``rounded'' up or down depending on who gets to play first). \\

These properties should make it clear that $*$ in particular is a very different object than the 
numbers we introduced previously.

Next, notice that in positions such as $\Uparrow*$ (and all subsequent games obtained 
by making longer ``corridors''), White has a much stronger incentive to play than 
White/Right: Black's play to block off the corridor, moving the game down to 0, while 
White's play moves the game down $\downarrow*$ in their favor. In effect, this means 
that Black does not need to respond urgently to White plays in corridors, until 
* is reached at which point they have increased incentive to respond to prevent White from connecting.
\\

Like for integers, using Theorems \ref{equivalence} and \ref{canonical} the infinitesimals 
we just defined can be treated interchangeably with corresponding Go positions. \\


Finally, to justify the name of infinitesimal, in nonstandard analysis, \textit{infinitesimals} are numbers
which are different from zero, but small than all positive real numbers, but larger than any negative
real numbers \cite{robinson2016non}. It can be easily verified that $* < \frac{1}{2^n}$ for $n \in \N$. 
This also holds for $\uparrow$, and in fact $n \uparrow$.


Conway also demonstrated in \cite[\S 0]{conway2000numbers} how to define any real number 
and \textit{ordinal number}
as a game (In some constructions, it is most natural to define infinitesimals and 
ordinals in relation to each other). 
However, while these do not show up in our analysis of Go endgames, 
it is interesting to note that combinatorial games provide a unified
framework for constructing infinitesimals, real numbers and ordinal numbers,
and that Go endgames motivated the discovery of this approach. (cf. \cite[Prologue]{conway2000numbers}).

\subsubsection{Playing with Numbers}

\begin{theorem}[Number Avoidance Theorem]
    \label{number_avoidance}
    Let $G$ be a game that is not equal to a number, and $x$ a number. Consider
    the game $G+x$. Then for both Left and Right making a move in $G$ is at 
    least as good as making a move in $x$.
\end{theorem}
    This means that in general players should not move onto a number unless 
    they have another option. This allows us to simplify games further. \\
    
    In terms of Go, some of the implications of this theorem are already known 
    intuitively to competent players. In \S 3.6.2 we showed that integers correspond
    to ``solid'' territory for one player. For instance, White should not want to 
    play in ``solid'' territory for Black since their stones can always be captured by
    Black's response. Black on the other hand should not play in their own territory, 
    since they would be eating into their own free moves (or points). 
\end{document}
